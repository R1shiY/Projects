% Options for packages loaded elsewhere
\PassOptionsToPackage{unicode}{hyperref}
\PassOptionsToPackage{hyphens}{url}
%
\documentclass[
]{article}
\usepackage{amsmath,amssymb}
\usepackage{iftex}
\ifPDFTeX
  \usepackage[T1]{fontenc}
  \usepackage[utf8]{inputenc}
  \usepackage{textcomp} % provide euro and other symbols
\else % if luatex or xetex
  \usepackage{unicode-math} % this also loads fontspec
  \defaultfontfeatures{Scale=MatchLowercase}
  \defaultfontfeatures[\rmfamily]{Ligatures=TeX,Scale=1}
\fi
\usepackage{lmodern}
\ifPDFTeX\else
  % xetex/luatex font selection
\fi
% Use upquote if available, for straight quotes in verbatim environments
\IfFileExists{upquote.sty}{\usepackage{upquote}}{}
\IfFileExists{microtype.sty}{% use microtype if available
  \usepackage[]{microtype}
  \UseMicrotypeSet[protrusion]{basicmath} % disable protrusion for tt fonts
}{}
\makeatletter
\@ifundefined{KOMAClassName}{% if non-KOMA class
  \IfFileExists{parskip.sty}{%
    \usepackage{parskip}
  }{% else
    \setlength{\parindent}{0pt}
    \setlength{\parskip}{6pt plus 2pt minus 1pt}}
}{% if KOMA class
  \KOMAoptions{parskip=half}}
\makeatother
\usepackage{xcolor}
\usepackage[margin=1in]{geometry}
\usepackage{color}
\usepackage{fancyvrb}
\newcommand{\VerbBar}{|}
\newcommand{\VERB}{\Verb[commandchars=\\\{\}]}
\DefineVerbatimEnvironment{Highlighting}{Verbatim}{commandchars=\\\{\}}
% Add ',fontsize=\small' for more characters per line
\usepackage{framed}
\definecolor{shadecolor}{RGB}{248,248,248}
\newenvironment{Shaded}{\begin{snugshade}}{\end{snugshade}}
\newcommand{\AlertTok}[1]{\textcolor[rgb]{0.94,0.16,0.16}{#1}}
\newcommand{\AnnotationTok}[1]{\textcolor[rgb]{0.56,0.35,0.01}{\textbf{\textit{#1}}}}
\newcommand{\AttributeTok}[1]{\textcolor[rgb]{0.13,0.29,0.53}{#1}}
\newcommand{\BaseNTok}[1]{\textcolor[rgb]{0.00,0.00,0.81}{#1}}
\newcommand{\BuiltInTok}[1]{#1}
\newcommand{\CharTok}[1]{\textcolor[rgb]{0.31,0.60,0.02}{#1}}
\newcommand{\CommentTok}[1]{\textcolor[rgb]{0.56,0.35,0.01}{\textit{#1}}}
\newcommand{\CommentVarTok}[1]{\textcolor[rgb]{0.56,0.35,0.01}{\textbf{\textit{#1}}}}
\newcommand{\ConstantTok}[1]{\textcolor[rgb]{0.56,0.35,0.01}{#1}}
\newcommand{\ControlFlowTok}[1]{\textcolor[rgb]{0.13,0.29,0.53}{\textbf{#1}}}
\newcommand{\DataTypeTok}[1]{\textcolor[rgb]{0.13,0.29,0.53}{#1}}
\newcommand{\DecValTok}[1]{\textcolor[rgb]{0.00,0.00,0.81}{#1}}
\newcommand{\DocumentationTok}[1]{\textcolor[rgb]{0.56,0.35,0.01}{\textbf{\textit{#1}}}}
\newcommand{\ErrorTok}[1]{\textcolor[rgb]{0.64,0.00,0.00}{\textbf{#1}}}
\newcommand{\ExtensionTok}[1]{#1}
\newcommand{\FloatTok}[1]{\textcolor[rgb]{0.00,0.00,0.81}{#1}}
\newcommand{\FunctionTok}[1]{\textcolor[rgb]{0.13,0.29,0.53}{\textbf{#1}}}
\newcommand{\ImportTok}[1]{#1}
\newcommand{\InformationTok}[1]{\textcolor[rgb]{0.56,0.35,0.01}{\textbf{\textit{#1}}}}
\newcommand{\KeywordTok}[1]{\textcolor[rgb]{0.13,0.29,0.53}{\textbf{#1}}}
\newcommand{\NormalTok}[1]{#1}
\newcommand{\OperatorTok}[1]{\textcolor[rgb]{0.81,0.36,0.00}{\textbf{#1}}}
\newcommand{\OtherTok}[1]{\textcolor[rgb]{0.56,0.35,0.01}{#1}}
\newcommand{\PreprocessorTok}[1]{\textcolor[rgb]{0.56,0.35,0.01}{\textit{#1}}}
\newcommand{\RegionMarkerTok}[1]{#1}
\newcommand{\SpecialCharTok}[1]{\textcolor[rgb]{0.81,0.36,0.00}{\textbf{#1}}}
\newcommand{\SpecialStringTok}[1]{\textcolor[rgb]{0.31,0.60,0.02}{#1}}
\newcommand{\StringTok}[1]{\textcolor[rgb]{0.31,0.60,0.02}{#1}}
\newcommand{\VariableTok}[1]{\textcolor[rgb]{0.00,0.00,0.00}{#1}}
\newcommand{\VerbatimStringTok}[1]{\textcolor[rgb]{0.31,0.60,0.02}{#1}}
\newcommand{\WarningTok}[1]{\textcolor[rgb]{0.56,0.35,0.01}{\textbf{\textit{#1}}}}
\usepackage{graphicx}
\makeatletter
\def\maxwidth{\ifdim\Gin@nat@width>\linewidth\linewidth\else\Gin@nat@width\fi}
\def\maxheight{\ifdim\Gin@nat@height>\textheight\textheight\else\Gin@nat@height\fi}
\makeatother
% Scale images if necessary, so that they will not overflow the page
% margins by default, and it is still possible to overwrite the defaults
% using explicit options in \includegraphics[width, height, ...]{}
\setkeys{Gin}{width=\maxwidth,height=\maxheight,keepaspectratio}
% Set default figure placement to htbp
\makeatletter
\def\fps@figure{htbp}
\makeatother
\setlength{\emergencystretch}{3em} % prevent overfull lines
\providecommand{\tightlist}{%
  \setlength{\itemsep}{0pt}\setlength{\parskip}{0pt}}
\setcounter{secnumdepth}{-\maxdimen} % remove section numbering
\ifLuaTeX
  \usepackage{selnolig}  % disable illegal ligatures
\fi
\usepackage{bookmark}
\IfFileExists{xurl.sty}{\usepackage{xurl}}{} % add URL line breaks if available
\urlstyle{same}
\hypersetup{
  hidelinks,
  pdfcreator={LaTeX via pandoc}}

\author{}
\date{\vspace{-2.5em}}

\begin{document}

\begin{Shaded}
\begin{Highlighting}[]
\CommentTok{\#What is the CO2 uptake in plants, given various factors affecting the plant, and what are the strength of these factors?}
\end{Highlighting}
\end{Shaded}

\begin{Shaded}
\begin{Highlighting}[]
\FunctionTok{data}\NormalTok{(}\StringTok{"CO2"}\NormalTok{)}

\FunctionTok{head}\NormalTok{(CO2)}
\end{Highlighting}
\end{Shaded}

\begin{verbatim}
##   Plant   Type  Treatment conc uptake
## 1   Qn1 Quebec nonchilled   95   16.0
## 2   Qn1 Quebec nonchilled  175   30.4
## 3   Qn1 Quebec nonchilled  250   34.8
## 4   Qn1 Quebec nonchilled  350   37.2
## 5   Qn1 Quebec nonchilled  500   35.3
## 6   Qn1 Quebec nonchilled  675   39.2
\end{verbatim}

\begin{Shaded}
\begin{Highlighting}[]
\FunctionTok{str}\NormalTok{(CO2)}
\end{Highlighting}
\end{Shaded}

\begin{verbatim}
## Classes 'nfnGroupedData', 'nfGroupedData', 'groupedData' and 'data.frame':   84 obs. of  5 variables:
##  $ Plant    : Ord.factor w/ 12 levels "Qn1"<"Qn2"<"Qn3"<..: 1 1 1 1 1 1 1 2 2 2 ...
##  $ Type     : Factor w/ 2 levels "Quebec","Mississippi": 1 1 1 1 1 1 1 1 1 1 ...
##  $ Treatment: Factor w/ 2 levels "nonchilled","chilled": 1 1 1 1 1 1 1 1 1 1 ...
##  $ conc     : num  95 175 250 350 500 675 1000 95 175 250 ...
##  $ uptake   : num  16 30.4 34.8 37.2 35.3 39.2 39.7 13.6 27.3 37.1 ...
##  - attr(*, "formula")=Class 'formula'  language uptake ~ conc | Plant
##   .. ..- attr(*, ".Environment")=<environment: R_EmptyEnv> 
##  - attr(*, "outer")=Class 'formula'  language ~Treatment * Type
##   .. ..- attr(*, ".Environment")=<environment: R_EmptyEnv> 
##  - attr(*, "labels")=List of 2
##   ..$ x: chr "Ambient carbon dioxide concentration"
##   ..$ y: chr "CO2 uptake rate"
##  - attr(*, "units")=List of 2
##   ..$ x: chr "(uL/L)"
##   ..$ y: chr "(umol/m^2 s)"
\end{verbatim}

\begin{Shaded}
\begin{Highlighting}[]
\FunctionTok{summary}\NormalTok{(CO2)}
\end{Highlighting}
\end{Shaded}

\begin{verbatim}
##      Plant             Type         Treatment       conc          uptake     
##  Qn1    : 7   Quebec     :42   nonchilled:42   Min.   :  95   Min.   : 7.70  
##  Qn2    : 7   Mississippi:42   chilled   :42   1st Qu.: 175   1st Qu.:17.90  
##  Qn3    : 7                                    Median : 350   Median :28.30  
##  Qc1    : 7                                    Mean   : 435   Mean   :27.21  
##  Qc3    : 7                                    3rd Qu.: 675   3rd Qu.:37.12  
##  Qc2    : 7                                    Max.   :1000   Max.   :45.50  
##  (Other):42
\end{verbatim}

\begin{Shaded}
\begin{Highlighting}[]
\FunctionTok{library}\NormalTok{(ggplot2)}

\FunctionTok{ggplot}\NormalTok{(CO2, }\FunctionTok{aes}\NormalTok{(}\AttributeTok{x =}\NormalTok{ conc, }\AttributeTok{y =}\NormalTok{ uptake, }\AttributeTok{color =}\NormalTok{ Treatment)) }\SpecialCharTok{+}
  \FunctionTok{geom\_point}\NormalTok{() }\SpecialCharTok{+}
  \FunctionTok{geom\_smooth}\NormalTok{(}\AttributeTok{method =} \StringTok{"lm"}\NormalTok{) }\SpecialCharTok{+}
  \FunctionTok{facet\_wrap}\NormalTok{(}\SpecialCharTok{\textasciitilde{}}\NormalTok{Type) }\SpecialCharTok{+}
  \FunctionTok{labs}\NormalTok{(}\AttributeTok{title =} \StringTok{"CO2 Uptake vs Concentration"}\NormalTok{, }\AttributeTok{x =} \StringTok{"Concentration"}\NormalTok{, }\AttributeTok{y =} \StringTok{"Uptake"}\NormalTok{)}
\end{Highlighting}
\end{Shaded}

\begin{verbatim}
## `geom_smooth()` using formula = 'y ~ x'
\end{verbatim}

\includegraphics{CO2-Uptake_files/figure-latex/unnamed-chunk-4-1.pdf}

\begin{Shaded}
\begin{Highlighting}[]
\FunctionTok{sum}\NormalTok{(}\FunctionTok{is.na}\NormalTok{(CO2))}
\end{Highlighting}
\end{Shaded}

\begin{verbatim}
## [1] 0
\end{verbatim}

\begin{Shaded}
\begin{Highlighting}[]
\NormalTok{CO2 }\OtherTok{\textless{}{-}} \FunctionTok{na.omit}\NormalTok{(CO2)}
\end{Highlighting}
\end{Shaded}

\begin{Shaded}
\begin{Highlighting}[]
\FunctionTok{library}\NormalTok{(caret)}
\end{Highlighting}
\end{Shaded}

\begin{verbatim}
## Loading required package: lattice
\end{verbatim}

\begin{Shaded}
\begin{Highlighting}[]
\FunctionTok{set.seed}\NormalTok{(}\DecValTok{123}\NormalTok{)}
\NormalTok{trainIndex }\OtherTok{\textless{}{-}} \FunctionTok{createDataPartition}\NormalTok{(CO2}\SpecialCharTok{$}\NormalTok{uptake, }\AttributeTok{p =} \FloatTok{0.8}\NormalTok{, }\AttributeTok{list =} \ConstantTok{FALSE}\NormalTok{, }\AttributeTok{times =} \DecValTok{1}\NormalTok{)}

\NormalTok{CO2Train }\OtherTok{\textless{}{-}}\NormalTok{ CO2[trainIndex, ]}
\NormalTok{CO2Test }\OtherTok{\textless{}{-}}\NormalTok{ CO2[}\SpecialCharTok{{-}}\NormalTok{trainIndex, ]}

\NormalTok{model }\OtherTok{\textless{}{-}} \FunctionTok{glm}\NormalTok{(uptake }\SpecialCharTok{\textasciitilde{}}\NormalTok{ conc }\SpecialCharTok{+}\NormalTok{ Treatment }\SpecialCharTok{+}\NormalTok{ Type, }\AttributeTok{data =}\NormalTok{ CO2Train)}
\end{Highlighting}
\end{Shaded}

\begin{Shaded}
\begin{Highlighting}[]
\NormalTok{train\_control }\OtherTok{\textless{}{-}} \FunctionTok{trainControl}\NormalTok{(}\AttributeTok{method =} \StringTok{"cv"}\NormalTok{, }\AttributeTok{number =} \DecValTok{10}\NormalTok{)}

\NormalTok{model\_cv }\OtherTok{\textless{}{-}} \FunctionTok{train}\NormalTok{(uptake }\SpecialCharTok{\textasciitilde{}}\NormalTok{ conc }\SpecialCharTok{+}\NormalTok{ Treatment }\SpecialCharTok{+}\NormalTok{ Type, }\AttributeTok{data =}\NormalTok{ CO2Train, }\AttributeTok{method =} \StringTok{"lm"}\NormalTok{, }\AttributeTok{trControl =}\NormalTok{ train\_control)}

\FunctionTok{summary}\NormalTok{(model\_cv)}
\end{Highlighting}
\end{Shaded}

\begin{verbatim}
## 
## Call:
## lm(formula = .outcome ~ ., data = dat)
## 
## Residuals:
##      Min       1Q   Median       3Q      Max 
## -15.7011  -2.8947   0.8369   4.3378  10.9539 
## 
## Coefficients:
##                    Estimate Std. Error t value Pr(>|t|)    
## (Intercept)       30.021502   1.677426  17.897  < 2e-16 ***
## conc               0.017680   0.002697   6.556 1.11e-08 ***
## Treatmentchilled  -7.295361   1.511916  -4.825 9.03e-06 ***
## TypeMississippi  -13.432816   1.507455  -8.911 8.09e-13 ***
## ---
## Signif. codes:  0 '***' 0.001 '**' 0.01 '*' 0.05 '.' 0.1 ' ' 1
## 
## Residual standard error: 6.193 on 64 degrees of freedom
## Multiple R-squared:  0.6761, Adjusted R-squared:  0.661 
## F-statistic: 44.54 on 3 and 64 DF,  p-value: 1.146e-15
\end{verbatim}

\begin{Shaded}
\begin{Highlighting}[]
\NormalTok{predictions }\OtherTok{\textless{}{-}} \FunctionTok{predict}\NormalTok{(model, CO2Test)}

\NormalTok{results }\OtherTok{\textless{}{-}} \FunctionTok{data.frame}\NormalTok{(}\AttributeTok{Actual =}\NormalTok{ CO2Test}\SpecialCharTok{$}\NormalTok{uptake, }\AttributeTok{Predicted =}\NormalTok{ predictions)}

\NormalTok{MAE }\OtherTok{\textless{}{-}} \FunctionTok{mean}\NormalTok{(}\FunctionTok{abs}\NormalTok{(results}\SpecialCharTok{$}\NormalTok{Actual }\SpecialCharTok{{-}}\NormalTok{ results}\SpecialCharTok{$}\NormalTok{Predicted))}
\NormalTok{RMSE }\OtherTok{\textless{}{-}} \FunctionTok{sqrt}\NormalTok{(}\FunctionTok{mean}\NormalTok{((results}\SpecialCharTok{$}\NormalTok{Actual }\SpecialCharTok{{-}}\NormalTok{ results}\SpecialCharTok{$}\NormalTok{Predicted)}\SpecialCharTok{\^{}}\DecValTok{2}\NormalTok{))}

\FunctionTok{cat}\NormalTok{(}\StringTok{"Mean Absolute Error: "}\NormalTok{, MAE, }\StringTok{"}\SpecialCharTok{\textbackslash{}n}\StringTok{"}\NormalTok{)}
\end{Highlighting}
\end{Shaded}

\begin{verbatim}
## Mean Absolute Error:  4.729453
\end{verbatim}

\begin{Shaded}
\begin{Highlighting}[]
\FunctionTok{cat}\NormalTok{(}\StringTok{"Root Mean Squared Error: "}\NormalTok{, RMSE, }\StringTok{"}\SpecialCharTok{\textbackslash{}n}\StringTok{"}\NormalTok{)}
\end{Highlighting}
\end{Shaded}

\begin{verbatim}
## Root Mean Squared Error:  6.28426
\end{verbatim}

\begin{Shaded}
\begin{Highlighting}[]
\FunctionTok{ggplot}\NormalTok{(results, }\FunctionTok{aes}\NormalTok{(}\AttributeTok{x =}\NormalTok{ Actual, }\AttributeTok{y =}\NormalTok{ Predicted)) }\SpecialCharTok{+}
  \FunctionTok{geom\_point}\NormalTok{() }\SpecialCharTok{+}
  \FunctionTok{geom\_abline}\NormalTok{(}\AttributeTok{slope =} \DecValTok{1}\NormalTok{, }\AttributeTok{intercept =} \DecValTok{0}\NormalTok{, }\AttributeTok{color =} \StringTok{"red"}\NormalTok{) }\SpecialCharTok{+}
  \FunctionTok{labs}\NormalTok{(}\AttributeTok{title =} \StringTok{"Actual vs Predicted CO2 Uptake"}\NormalTok{, }\AttributeTok{x =} \StringTok{"Actual Uptake"}\NormalTok{, }\AttributeTok{y =} \StringTok{"Predicted Uptake"}\NormalTok{)}
\end{Highlighting}
\end{Shaded}

\includegraphics{CO2-Uptake_files/figure-latex/unnamed-chunk-8-1.pdf}

\begin{Shaded}
\begin{Highlighting}[]
\CommentTok{\#Summary}

\CommentTok{\#The linear regression model indicates:}

\CommentTok{\#All predictors (conc, Treatment, and Type) are highly significant.}
\CommentTok{\#The intercept suggests that the baseline uptake (with all predictors at zero) is 30.021502.}
\CommentTok{\#The concentration of CO2 has a positive effect on uptake.}
\CommentTok{\#The "chilled" treatment and "Mississippi" type both have negative effects on uptake.}
\CommentTok{\#The model explains approximately 67.61\% of the variance in CO2 uptake, and the model fit is statistically significant.}
\CommentTok{\#The results provide strong evidence that the concentration, treatment, and type significantly affect CO2 uptake.}
\end{Highlighting}
\end{Shaded}


\end{document}
